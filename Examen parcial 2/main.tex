\documentclass[letterpaper]{article}
\usepackage[utf8]{inputenc}
\usepackage{amsmath,amsfonts,amssymb} 
\usepackage{ mathrsfs }
\usepackage{graphicx}
\usepackage[usenames]{xcolor}
\usepackage{array}
\usepackage{enumitem}
\usepackage[spanish,es-nodecimaldot]{babel}
\usepackage{hyperref}
\usepackage{ragged2e}

\title{\includegraphics[scale=0.2]{Logo.jpg}\\
División de Arquitectura, Diseño e Ingenierías\\
\colorbox{black}{. \hspace{2.45cm} \textcolor{white}{\textbf{Examen parcial 2 - Informática I}} \hspace{2.45cm} .}}
\author{\textcolor{red}{\textbf{Escriba aquí su nombre}}}
\date{\today}

\voffset -3cm
\evensidemargin 0.8cm
\oddsidemargin -0.3cm
\textwidth 16.0cm
\textheight 24cm
\headsep 0pt

\begin{document}
\maketitle

\textbf{Edite este archivo \LaTeX respondiendo a cada pregunta correctamente} (muy importante esto último). \textbf{Justifique cada respuesta} (con procedimiento).

Escriba su nombre y mande los archivos fuente y el pdf al correo \href{mailto:arellano@suscience.com}{arellano@suscience.com}.

\begin{enumerate}
    \item ¿Cuál es la memoria volátil de lectura--escritura? puede ser dinámica o estática.
    
    
    
    
    \item ¿Cuál es la memoria de sólo lectura no--volátil?
    
    
    
    
    \item ¿Cuál es la memoria de acceso rápido, más que la RAM pero menos que los registros de la unidad de control?
    
    
    
    
    \item Una computadora tiene 256 MB de memoria, ¿Cuántos bits se necesitan para asignar una dirección a cada palabra si ésta mide 4 bytes?
    
    
    
    
    \item Suponga que un programa comienza en la dirección 208. ¿En qué dirección finaliza el siguiente código máquina?
    
    \begin{tabular}{|lcc|}
    \hline
        Copiar  & R1 & @2000\\
        Copiar & R2 & @2001\\
        Multiplicar & R1 & \\
        Copiar & R2 & @2002\\
        Sumar & R3 & \\
        Copiar & @2003 & R3\\
        FIN&&\\
    \hline
    \end{tabular}
    
    
    
    
    \item Escriba el pseudocódigo para ejecutar la operación factorial de un número dado como entrada.
    \begin{itemize}
        \item El factorial de $N$, expresado como $N!$, se define como la multiplicación sucesiva de los números anteriores. Por ejemplo:
        $$5!=5\times4\times3\times2\times1=120.$$
    \end{itemize}
    
    
    
    
    \item Realice las siguientes operaciones:
    \begin{eqnarray*}
    11011011.1&\times\\
    \underline{\hspace{5mm}11001\;\;\;}&\\
    \mathtt{{\color{gray}Escriba\;su\;respuesta\;aqui}}\\
    \mathtt{{\color{gray}Puede\;agregar\;tantos\;renglones\;sean\;necesarios}}
    \end{eqnarray*}
    
    
    \begin{eqnarray*}
    1000110&-\\
    \underline{\hspace{5mm}11011}&\\
    \mathtt{{\color{gray}Escriba\;su\;respuesta\;aqui}}
    \end{eqnarray*}
    
    
    
    
    \item Use compuertas lógicas para representar la siguiente operación de 1 bit por hardware:
    $$
    a+\overline{b+c}-(d+a\times b)
    $$
    {\color{gray}Use las compuertas adjuntas en el repositorio para crear un circuito y agregar la imagen al documento. Cualquier duda me pueden escribir.}
    
    
    
    
    \item Realice la siguiente operación lógica:
    
    \centering
     [a \textbf{AND}(a \textbf{OR} b)]\textbf{OR}(a \textbf{AND} b)
     
     \justify
     con a=110 y b=011.
     
     
     
\end{enumerate}
\end{document}
